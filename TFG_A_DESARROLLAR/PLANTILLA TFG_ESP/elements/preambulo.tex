% -------------------------------
% Plantilla del TFG en la ESIIAB-UCLM. Versión 0.0 (Preliminar)
%
% Luis de la Ossa
% 
% Compilar con XeLaTeX 
% -------------------------------


% -------------------------------
% Segunda portada
% -------------------------------
\cleardoublepage
\setcounter{page}{1} \null
\begin{textblock*}{3cm}(3cm,2cm) 
\begin{flushleft}
\includegraphics[height=2cm]{figs/logouclm.png}
\end{flushleft}
\end{textblock*}


\begin{textblock*}{2.5cm}(15.95cm,2.45cm) 
\begin{flushright}
\includegraphics[height=1.4cm]{figs/euroinf.jpg}
\end{flushright}
\end{textblock*}

\begin{textblock*}{2cm}(14cm,2cm)  % Corrige la posición porque la imagen tiene margen
\begin{flushright}
\includegraphics[height=2cm]{figs/logoesii.png}
\end{flushright}
\end{textblock*}



\begin{textblock*}{\textwidth}(3cm,8cm) 
\begin{center}\doublespacing 
{\fontsize{18pt}{4pt}\selectfont \bft{TRABAJO FIN DE GRADO}}

\medskip
{\fontsize{18pt}{4pt}\selectfont Grado en Ingeniería Informática}

\medskip
{\fontsize{16pt}{4pt}\selectfont \espec}
\end{center}
\end{textblock*}


\begin{textblock*}{\textwidth}(3cm,14cm) 
\begin{center}\doublespacing 
{\fontsize{22pt}{4pt}\selectfont \bft{\titulo}}
\end{center}
\end{textblock*}


\begin{textblock*}{\textwidth}(3cm,20.5cm) 
\begin{flushleft}\doublespacing
{\fontsize{14pt}{4pt}\selectfont \bft{Autor:} \autor}

{\fontsize{14pt}{4pt}\selectfont \bft{Tutor:} \director}

{\fontsize{14pt}{4pt}\selectfont \bft{Co-Tutor:} \codirector}
\end{flushleft}
\end{textblock*}


\begin{textblock*}{\linewidth}(3cm,25cm) 
\begin{flushright}
{\fontsize{14pt}{4pt}\selectfont \fecha}
\end{flushright}
\end{textblock*}


% -------------------------------
% Dedicatoria
% -------------------------------
\cleardoublepage
\thispagestyle{empty}

\vspace*{9cm}  
\begin{flushright} \em 
A todas las personas que me han ayudado a llegar hasta aquí, sobre todo a mi familia y a mis amigos
\end{flushright}


% -------------------------------
% Declaración de autoría
% -------------------------------
\cleardoublepage
\thispagestyle{plain}
\begin{center}
\Large{\bft{Declaración de autoría}}
\end{center}
\vskip1cm

Yo, \autor con DNI \dni, declaro que soy el único autor del trabajo fin de grado titulado "Modelado del protocolo RDMA en un simulador de redes de interconexión basado en OMNeT++", que el citado trabajo no infringe las leyes en vigor sobre propiedad intelectual, y que todo el material no original contenido en dicho trabajo está apropiadamente atribuido a sus legítimos autores.

\vspace*{2cm}
\begin{center}
Albacete, a \fecha

\vskip3cm

Fdo.: \autor
\end{center}


% -------------------------------
% Resumen
% -------------------------------
\cleardoublepage
\thispagestyle{plain}
\begin{center}
\Large{\bft{Resumen}}
\end{center}
\vskip1cm

RDMA es un concepto que usa DMA, es decir, permite a los dispositivos de la placa base de la computadora enviar datos directamente a memoria cuando dos o más computadoras se comunican, accediendo a la memoria de otro host directamente desde la memoria de un host.\\

Emplearemos la herramienta OMNeT++ que es una biblioteca y un marco de simulación C ++ extensible, modular y basado en componentes, principalmente para la construcción de simuladores de red. Aunque OMNeT ++ no es un simulador de red en sí mismo, ha ganado una gran popularidad como plataforma de simulación de red en la comunidad científica pues proporciona una arquitectura de componentes para modelos. Los componentes (módulos) se programan en C ++ y luego se ensamblan en componentes y modelos más grandes utilizando un lenguaje de alto nivel (NED). La reutilización de modelos es gratuita. OMNeT++ tiene un amplio soporte de GUI y, debido a su arquitectura modular, el kernel de simulación (y los modelos) se pueden integrar fácilmente en sus aplicaciones. \cite{Omnet} \\

Teniendo en cuenta lo anterior, este TFG tratará de modelar RDMA en OMNeT++, pues la herramienta no tiene esta funcionalidad y compararlo con TCP/IP con el objetivo de demostrar que tiene un mejor funcionamiento


% -------------------------------
% Agradecimientos
% -------------------------------
\cleardoublepage
\thispagestyle{plain}
\begin{center}
\Large{\bft{Agradecimientos}}
\end{center}
\vskip1cm
Me gustaría agradecer a mis padres, por haberme pagado la carrera y haberme apoyado en todo momento con todo lo relacionado a los estudios, a todos las personas que he conocido a lo largo de mi vida, tanto a aquellas que te ayudan a crecer estando a tu lado, como a aquellas que suponen un reto para ti y te ayudan a superar momentos y superarte a tí mismo. Por supuesto, también agradecer a todos los profesores que me han ayudado a llegar hasta aquí y, sobre todo, a Jesús y a Paco, por confiar en mí y proponerme la realización de este TFG.