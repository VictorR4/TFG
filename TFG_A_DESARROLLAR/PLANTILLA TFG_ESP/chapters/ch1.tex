\chapter{Introducción}
\lipsum[1-5]
\section{Motivación}
\lipsum[1-5]
\section{Objetivos}
El objetivo de este Trabajo de Fin de Grado es el de modelar el protocolo de transporte RDMA en el modelo INET del simulador Omnet++ y realizar las comparaciones necesarias para demostrar la diferencia de rendimiento de una red según usemos RDMA o TCP/IP. Para ello vamos a seguir los siguientes pasos.
\begin{enumerate}
	\item Conocer el estado del arte de las redes de interconexión en general, y del protocolo RDMA en particular.
	\item Estudiar las herramientas de simulación existentes basadas en OMNeT++, como el modelo INET que ya dispone del protocolo de transporte TCP/IP.
	\item Modelar el protocolo RDMA en el modelo INET de OMNeT++.
	\item Realizar experimentos de simulación comparando varias configuraciones de red de interconexión que usen TCP/IP y RDMA, y hacer un estudio comparativo de las prestaciones 
\end{enumerate}
\section{Estructura de la memoria}
La memoria esta organizada de la siguiente manera: 

\begin{itemize}
	\item \textbf{Introducción:}
	\item \textbf{Antecedentes y estado de la cuestión:} se enfoca en ofrecer una introducción sobre la redes de interconexión. En ella se informa sobre las diferentes topologías de redes de interconexión y características específicas y generales de ellas, también se indican los componentes de una red de interconexión, la arquitectura interna del switch, los diferentes tipos de encaminamiento de los mensajes por la red y problemas relacionados con ellos y control de flujo. También nos enfocaremos en el que será la prioridad de este trabajo, el protocolo de transporte RDMA. Por último, se describen las ideas principales del simulador Omnet++.
	\item \textbf{Metodología y desarrollo:}
	\item \textbf{Experimentos y resultados:}
	\item \textbf{Conclusiones:}
\end{itemize}